\documentclass[11pt,a4paper,parskip=half]{scrartcl}

\usepackage[english]{babel}

\usepackage[utf8]{inputenc}
\usepackage[T1]{fontenc}
\usepackage{graphicx}
\graphicspath{ {./images/} }

\usepackage{settings}


\usepackage{csquotes}
\MakeOuterQuote{"}
\usepackage{fontawesome}

\usepackage{minted}
\usepackage{soul}
% https://v2.overleaf.com/project/5af2cde849f63e7a51037537

\title{Product Management at Overleaf} 

\begin{document}
\maketitle
\vspace{-1em}
\textit{This document is for internal use only. Beware of the big bad wolf. Last Updated \today.}

\section*{Product Management} 

Write something clever about product management?

\tableofcontents
\newpage 

\section{Introduction}
Overleaf’s integration with Sharelatex and the rapid pace of growing B2C customers is being successfully supported by an awesome team, where knowledge and work are shared between sales, support, development, and management.
However, despite the successful evolution of the company — at all levels —, the challenges ahead for sustainable growth require improvements on some existing product management processes. Until now, implemented systems, processes and workflows are effective, but not as efficient as they could be with the available resources — team and software/hardware.
This document will uncover several key changes to some internal processes, which can be vastly improved. These evolutionary changes will be carried away at a new interaction level — the product level — and have the main goal of creating a more efficient process across teams, which will enable us to continue delivering excellent quality for the next 10 million users.

\section{Finances and Organisation}
Notes

\section{Community and Support}

\section{Marketing and Sales}

\section{Software}

\subsection{Internal Organisation}
Development work force is split into three teams: Team Aqua, Team Magma and Team Ops. There is no clear area separation, except for the Ops team. Aqua and Magma have members covering the full development stack, from frontend to integrations. There is also no clear division of work in sub-products or features, but a natural separation of interests and assigned tasks is starting to push this division. teying 

\subsection{Identified Issues and Needs}
\subsubsection{Academia, Universities and Enterprise}
\begin{itemize}
\item Make sure users lose access when they leave university
\end{itemize}
\subsubsection{Access Management}
\begin{itemize}
\item SSO
\item leaver policy: revoke access when users leave university (can this be solved with institutional SSO??
\end{itemize}
\subsubsection{Editor}
\begin{itemize}
\item compilation and auto-compile (too fast)
\item Large images are problematic
\item hidden features: default compile mode
\item timeout errors
\item accessibility
\item be excellent on editor (90 of contact with users): dashboard, history view...
\item performance still an issue, albeit not relevant
\item Stuff missing from v1
\item Why so many missing features?
\end{itemize}

\subsection{Product Management}

Managing the product for the development team.
\begin{itemize}
\item Lack of proper awareness regarding past, present and future
\item Everyone should be able to know, at any given point in time
\item what was done, and why
\item what is being done, and why
\item what is coming next, and why
\end{itemize}

\section{Product Management}
\subsection{Planning Ahead}
Although there’s a rough company roadmap, sketched in Miro, there’s no clear structure on the path to move forward: we have ideas, we have a (huge) number of features and requests, but there’s is nothing — written or documented — connecting the dots and setting the path to be taken by the company. 
From a product-oriented perspective, there needs to be a clear journey from what we are developing today to what we want to achieve in the near-, medium- and long-term. As such, there should at least be a quarterly review of the company’s objectives, to set the tone for the months. 
\subsubsection{OKRs vs Others}
There are several alternatives for cross-department planning. We can move as we are, or we can start delivering accurate product roadmaps for short-term needs. 

\subsubsection{Quarterly Themes}

A simple yet effective alternative is focused around quarterly themes. Each quarter, the management and product teams define a theme for the quarter. This can be a big feature or, in product lingo, one or more Epics. There should be themes prepared in advance for at least one year. The detail and specificity of each quarter should be very high on the next quarter and very loose on the far ahead future. 
The themes for the quarter should be concise: a descriptive paragraph, key features to be implemented and associated success metrics.

\subsubsection{OKRs}
OKRs have gained immense popularity in the last couple of years due to their adoption by several high-profile companies. 
OKRs are a combination of Objectives and Key Results. Each OKR has one clear objective and 3 key results. The objective should be concise and qualitative whereas the key results must be measurable.
The adoption of OKRs should cascade within the company. This means that there should be company-wide annual OKRs, which would help define department-wide annual OKRs, which would, in turn, set the tone for internal quarterly OKRs… 
OKRs should be reviewed weekly and progress tracked accurately on the measurable metrics (up, down, confidence on meeting expectations).

OKR-related links:
\begin{itemize}
    \item https://rework.withgoogle.com/guides/set-goals-with-okrs/steps/introduction/
    \item https://weekdone.com/resources/objectives-key-results
    \item https://weekdone.com/ebook/okr-goal-setting-guide-template
    \item https://www.perdoo.com/okr/
\end{itemize}

\subsubsection{Daily Operations}
There needs to be a central “source of truth”, shared across all levels of the company, that showcases where we stand and what’s coming next. This will be essential to get everyone on the same page about ongoing developments, upcoming features and medium- to long-term roadmap. 
\subsubsection{Sample Scenarios}
Feature Release
Release new features, specially the big ones, has to be a coordinated effort between several company departments. Such new tool would help us plan and synchronise releases, and coordinate the associated tasks from each department. 
For instance, to release feature X on date Y, we have:
\begin{itemize}
    \item Development team
    \begin{itemize}
        \item Keep the development roadmap updated with the realistic deployment timeline.
        \item Deploy the release online after proper testing and validation.
    \end{itemize}
    \item Marketing
    \begin{itemize}
        \item Prepare marketing materials (content, publicity media, …) before date Y.
        \item Publicise feature through Overleaf’s communication channels..
    \end{itemize}
     \item Marketing
    \begin{itemize}
        \item Shares feature availability to clients/users who requested it, sends materials for internal dissemination.
    \end{itemize}
\end{itemize}{}


\subsubsection{Example}
The development team got a  feature request for adding a list of subscribers (and their individual status) to the institutions hub. This a small feature and was quickly prioritised and implemented. After iterating the implementation based on the team’s feedback, the feature was ready for dissemination. Alas, there was no plan in place to showcase this new feature. This is something that should be set up from the start, where the development, marketing and sales teams worked in tandem to get everyone on the same page about this feature. Development team implements, marketing promotes, sales disseminates to users who requested this feature.

Product Roadmap
The long-term product roadmap is properly discussed between all company departments. This new tool would help prioritise feature development, simplifying the construction of a proper product development roadmap. 
This would have to take into account needs from all departments:
Sales
“We need to increase revenue 10 in two quarters. What features are we releasing to help convince new users?”
Institutional SSO is a good example for this!
Marketing
“Big event Z is coming in two months, what will be available by then that we can promote to improve user acquisition?”
Support
“There’s been a growth in complaints about feature W, when will it be solved?”
TexLive upgrade is a good example of this!
Development
“We only have bandwidth to develop three big features next quarter, what are your priorities?”


\end{document}
